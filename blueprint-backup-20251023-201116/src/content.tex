% Main mathematical content for Ricci Flow formalization

\chapter{Introduction}

This project aims to formalize the theory of Ricci Flow in Lean 4, building upon the Mathlib library. The Ricci Flow is a powerful tool in differential geometry, introduced by Richard Hamilton and famously used by Grigori Perelman in his proof of the Poincaré conjecture.

The formalization is organized into several modules, each handling different aspects of the theory:
\begin{itemize}
\item Basic definitions and topological foundations
\item Riemannian manifolds and metric tensors
\item Ricci curvature tensors
\item The Ricci Flow equation and existence theorems
\end{itemize}

\chapter{Basic Definitions}
\label{chap:basic}

\begin{definition}[Manifold Type]
\label{def:manifold-type}
\lean{RicciFlow.Basic}
\leanok
We work with a type $M$ equipped with a topological space structure and a charted space structure modeled on $\R$.
\end{definition}

\begin{lemma}[Positive multiplication]
\label{lem:pos-mul-pos}
\lean{RicciFlow.pos_mul_pos}
\leanok
\uses{def:manifold-type}
The product of two positive real numbers is positive.
\end{lemma}

\begin{lemma}[Square positivity]
\label{lem:square-pos}
\lean{RicciFlow.square_pos_of_ne_zero}
\leanok
The square of a nonzero real number is strictly positive.
\end{lemma}

\begin{lemma}[Existence of positive real]
\label{lem:exists-pos-real}
\lean{RicciFlow.exists_pos_real}
\leanok
There exists a positive real number.
\end{lemma}

\begin{lemma}[Inverse positivity]
\label{lem:inv-pos}
\lean{RicciFlow.inv_pos_of_pos}
\leanok
The inverse of a positive real number is positive.
\end{lemma}

\begin{lemma}[Continuous at characterization]
\label{lem:continuous-at-iff}
\lean{RicciFlow.continuousAt_iff_continuousWithinAt}
\leanok
\uses{def:manifold-type}
A function is continuous at a point if and only if it is continuous within the universal set at that point.
\end{lemma}

\chapter{Riemannian Manifolds}
\label{chap:riemannian}

\begin{definition}[Riemannian Metric]
\label{def:riemannian-metric}
\lean{RicciFlow.RiemannianMetric}
\uses{def:manifold-type}
A Riemannian metric on a manifold $M$ is a smooth, symmetric, positive-definite $(0,2)$-tensor field.

In the current simplified implementation, we use type parameters $M$ (the manifold) and $V$ (representing the abstract tangent space). The metric at each point $x \in M$ gives a bilinear form $g_x : V \times V \to \R$ satisfying:
\begin{enumerate}
\item \textbf{Symmetry}: $g_x(v, w) = g_x(w, v)$ for all $v, w \in V$
\item \textbf{Positive-definiteness}: $g_x(v, v) > 0$ for all $v \neq 0$
\end{enumerate}
\end{definition}

\begin{remark}
The current implementation provides the structure definition with type parameter $V$ for the tangent space. Future phases will:
\begin{itemize}
\item Phase 2: Use dependent types for concrete tangent spaces
\item Phase 3: Implement as smooth tensor fields $\operatorname{SmoothSection}(\operatorname{Sym}^2 T^*M)$
\end{itemize}
\end{remark}

\begin{definition}[Inner Product]
\label{def:inner-product}
\lean{RicciFlow.innerProduct}
\uses{def:riemannian-metric}
The inner product $\inner{v}{w}_x$ at point $x \in M$ is given by the metric: $\inner{v}{w}_x = g_x(v, w)$.
\end{definition}

\begin{definition}[Norm squared]
\label{def:norm-squared}
\lean{RicciFlow.normSq}
\uses{def:inner-product}
The squared norm of a vector $v$ at point $x$ is $\norm{v}^2_x = g_x(v, v) = \inner{v}{v}_x$.
\end{definition}

\begin{lemma}[Inner product symmetry]
\label{lem:inner-product-symm}
\lean{RicciFlow.innerProduct_symm}
\leanok
\uses{def:inner-product}
The inner product is symmetric: $\inner{v}{w}_x = \inner{w}{v}_x$.
\end{lemma}

\begin{lemma}[Norm positivity]
\label{lem:norm-sq-pos}
\lean{RicciFlow.normSq_pos}
\leanok
\uses{def:norm-squared}
For any nonzero vector $v \in V$, we have $\norm{v}^2_x > 0$.
\end{lemma}

\chapter{Ricci Curvature}
\label{chap:ricci}

\begin{definition}[Ricci Tensor]
\label{def:ricci-tensor}
\lean{RicciFlow.RicciTensor}
\uses{def:riemannian-metric}
The Ricci curvature tensor $\Ric$ is obtained by contracting the Riemann curvature tensor:
\[
\Ric_{ij} = R^k_{ikj}
\]
where $R^l_{ijk}$ are the components of the Riemann curvature tensor.
\end{definition}

\begin{remark}
The current simplified implementation stores only the trace value (scalar curvature). A complete implementation should store all components $\Ric_{ij}$ of the Ricci tensor.
\end{remark}

\begin{definition}[Scalar Curvature]
\label{def:scalar-curvature}
\lean{RicciFlow.scalarCurvature}
\uses{def:ricci-tensor}
The scalar curvature $\Scal$ is the trace of the Ricci tensor with respect to the metric:
\[
\Scal = g^{ij} \Ric_{ij}
\]
where $g^{ij}$ denotes the inverse metric tensor.

\textbf{Geometric interpretation}:
\begin{itemize}
\item $\Scal > 0$: The manifold curves inward (like a sphere)
\item $\Scal < 0$: The manifold curves outward (like hyperbolic space)
\item $\Scal = 0$: The manifold is locally flat
\end{itemize}
\end{definition}

\begin{lemma}[Scalar curvature extraction]
\label{lem:scalar-curvature-eq}
\lean{RicciFlow.scalarCurvature_eq_traceValue}
\leanok
\uses{def:scalar-curvature}
In the current implementation, the scalar curvature equals the stored trace value.
\end{lemma}

\chapter{Ricci Flow}
\label{chap:flow}

\begin{definition}[Ricci Flow Equation]
\label{def:ricci-flow-equation}
\lean{RicciFlow.ricci_flow_equation}
\uses{def:riemannian-metric, def:ricci-tensor}
The Ricci Flow is the geometric evolution equation:
\[
\frac{\partial g}{\partial t} = -2 \Ric(g)
\]
where $g(t)$ is a time-dependent family of Riemannian metrics and $\Ric(g)$ is the Ricci curvature tensor of $g$.
\end{definition}

\begin{theorem}[Short-Time Existence]
\label{thm:short-time-existence}
\lean{RicciFlow.short_time_existence}
\uses{def:ricci-flow-equation}
Let $(M, g_0)$ be a compact Riemannian manifold. Then there exists a time $T > 0$ and a smooth family of metrics $g(t)$ for $t \in [0, T)$ such that:
\begin{enumerate}
\item $g(0) = g_0$ (initial condition)
\item $g(t)$ satisfies the Ricci Flow equation for all $t \in (0, T)$
\end{enumerate}
\end{theorem}

\begin{proof}
This is a deep result in geometric analysis, relying on the theory of parabolic PDEs. The proof involves:
\begin{itemize}
\item Showing that the Ricci Flow equation is a weakly parabolic system
\item Applying short-time existence theory for parabolic systems
\item Establishing appropriate estimates to ensure smoothness
\end{itemize}

In the Lean formalization, this theorem is currently stated with \texttt{sorry}. A complete formalization would require developing substantial PDE theory in Lean.
\end{proof}

\chapter{Future Directions}

\section{Immediate Goals}
\begin{itemize}
\item Complete the definition of Riemannian metrics with smoothness and positive-definiteness conditions
\item Implement the Ricci tensor using Mathlib's connection and curvature theory
\item Prove basic properties of scalar curvature
\end{itemize}

\section{Long-Term Goals}
\begin{itemize}
\item Formalize Hamilton's maximum principle for the Ricci Flow
\item Prove convergence results for special cases (e.g., spheres)
\item Develop the theory of Ricci solitons
\end{itemize}
