\chapter{Introduction}

This blueprint documents the formalization of Ricci Flow theory in Lean 4.
Ricci Flow is a fundamental geometric evolution equation introduced by Richard Hamilton,
which has profound applications including Perelman's proof of the Poincaré conjecture.

\chapter{Basic Lemmas}
\label{chap:basic}

This chapter contains fundamental lemmas about real numbers and topology.

\begin{lemma}[Positive Multiplication]
\label{lem:pos_mul_pos}
\lean{RicciFlow.pos_mul_pos}
\leanok
For positive real numbers $a > 0$ and $b > 0$, their product $a \cdot b > 0$.
\end{lemma}

\begin{proof}
\leanok
Direct application of Mathlib's \texttt{mul\_pos}.
\end{proof}

\begin{lemma}[Square Positive of Nonzero]
\label{lem:square_pos_of_ne_zero}
\lean{RicciFlow.square_pos_of_ne_zero}
\leanok
For any nonzero real number $x \neq 0$, we have $x^2 > 0$.
\end{lemma}

\begin{proof}
\leanok
Uses \texttt{sq\_pos\_of\_ne\_zero} from Mathlib.
\end{proof}

\begin{lemma}[Existence of Positive Real]
\label{lem:exists_pos_real}
\lean{RicciFlow.exists_pos_real}
\leanok
There exists a positive real number.
\end{lemma}

\begin{proof}
\leanok
We construct $1$ as a witness.
\end{proof}

\begin{lemma}[Inverse Positive of Positive]
\label{lem:inv_pos_of_pos}
\lean{RicciFlow.inv_pos_of_pos}
\leanok
For $x > 0$, we have $x^{-1} > 0$.
\end{lemma}

\begin{proof}
\leanok
Uses \texttt{inv\_pos} from Mathlib.
\end{proof}

\begin{lemma}[Continuous At iff Continuous Within At]
\label{lem:continuousAt_iff}
\lean{RicciFlow.continuousAt_iff_continuousWithinAt}
\leanok
Continuity at a point is equivalent to continuity within the universal set.
\end{lemma}

\begin{proof}
\leanok
Uses \texttt{continuousWithinAt\_univ} from Mathlib.
\end{proof}

\chapter{Riemannian Manifolds}
\label{chap:riemannian}

\begin{definition}[Riemannian Metric]
\label{def:riemannian_metric}
\lean{RicciFlow.RiemannianMetric}
\leanok
\uses{lem:pos_mul_pos, lem:square_pos_of_ne_zero}
A Riemannian metric on a manifold $M$ is a smooth positive-definite symmetric bilinear form on each tangent space.

For each point $x \in M$ and tangent vectors $v, w \in T_xM$:
\begin{itemize}
\item \textbf{Symmetry}: $g_x(v, w) = g_x(w, v)$
\item \textbf{Positive-definiteness}: $g_x(v, v) > 0$ for all $v \neq 0$
\end{itemize}
\end{definition}

\begin{definition}[Inner Product]
\label{def:inner_product}
\lean{RicciFlow.innerProduct}
\leanok
\uses{def:riemannian_metric}
The inner product of tangent vectors $v, w \in T_xM$ is defined as $\langle v, w \rangle_x = g_x(v, w)$.
\end{definition}

\begin{definition}[Norm Squared]
\label{def:norm_sq}
\lean{RicciFlow.normSq}
\leanok
\uses{def:riemannian_metric}
The squared norm of a tangent vector $v \in T_xM$ is $\|v\|^2 = g_x(v, v)$.
\end{definition}

\begin{lemma}[Inner Product Symmetry]
\label{lem:inner_product_symm}
\lean{RicciFlow.innerProduct_symm}
\leanok
\uses{def:inner_product}
For any tangent vectors $v, w$ at point $x$: $\langle v, w \rangle_x = \langle w, v \rangle_x$.
\end{lemma}

\begin{proof}
\leanok
\uses{def:riemannian_metric}
Follows from the symmetry axiom of the Riemannian metric.
\end{proof}

\begin{lemma}[Norm Squared Positivity]
\label{lem:norm_sq_pos}
\lean{RicciFlow.normSq_pos}
\leanok
\uses{def:norm_sq}
For any nonzero tangent vector $v$ at point $x$: $\|v\|^2 > 0$.
\end{lemma}

\begin{proof}
\leanok
\uses{def:riemannian_metric}
Follows from the positive-definiteness axiom.
\end{proof}

\chapter{Ricci Curvature}
\label{chap:ricci}

\begin{definition}[Ricci Tensor]
\label{def:ricci_tensor}
\lean{RicciFlow.RicciTensor}
\leanok
\uses{def:riemannian_metric}
The Ricci curvature tensor $\mathrm{Ric}_{ij}$ is obtained by contracting the Riemann curvature tensor:
\[ \mathrm{Ric}_{ij} = R^k_{ikj} = g^{kl} R_{likj} \]
\end{definition}

\begin{definition}[Scalar Curvature]
\label{def:scalar_curvature}
\lean{RicciFlow.scalarCurvature}
\leanok
\uses{def:ricci_tensor, lem:inv_pos_of_pos}
The scalar curvature is the trace of the Ricci tensor: $R = g^{ij} \mathrm{Ric}_{ij}$.

\textbf{Geometric meaning}:
\begin{itemize}
\item $R > 0$: positive curvature (sphere-like)
\item $R < 0$: negative curvature (hyperbolic)
\item $R = 0$: flat (Euclidean)
\end{itemize}
\end{definition}

\begin{lemma}[Scalar Curvature Computation]
\label{lem:scalar_curvature_eq}
\lean{RicciFlow.scalarCurvature_eq_traceValue}
\leanok
\uses{def:scalar_curvature}
The scalar curvature equals the trace value: $R = \mathrm{Ric}.\mathrm{traceValue}$.
\end{lemma}

\begin{proof}
\leanok
Definitional in our current implementation.
\end{proof}

\chapter{Ricci Flow}
\label{chap:flow}

\begin{definition}[Ricci Flow Equation]
\label{def:ricci_flow_equation}
\lean{RicciFlow.ricci_flow_equation}
\uses{def:riemannian_metric, def:ricci_tensor}
The Ricci flow is the geometric evolution equation:
\[ \frac{\partial g_{ij}}{\partial t} = -2 \mathrm{Ric}_{ij} \]

This equation deforms the metric to make the curvature more uniform.
\end{definition}

\begin{theorem}[Short-Time Existence]
\label{thm:short_time_existence}
\lean{RicciFlow.short_time_existence}
\uses{def:ricci_flow_equation, def:scalar_curvature, lem:exists_pos_real}
For any smooth compact Riemannian manifold $(M, g_0)$, there exists $T > 0$ and a smooth solution $g(t)$ to the Ricci flow for $t \in [0, T)$ with initial condition $g(0) = g_0$.
\end{theorem}

\begin{proof}
Hamilton's fundamental theorem (1982). The proof uses:
\begin{enumerate}
\item DeTurck's trick to break diffeomorphism invariance
\item Parabolic PDE theory for short-time existence
\item Pullback to obtain the Ricci flow solution
\end{enumerate}

Currently formalized as \texttt{sorry}.
\end{proof}

\chapter{Future Directions}

\section{Planned Extensions}

\begin{itemize}
\item Complete formalization of short-time existence
\item Maximum principles for Ricci flow
\item Normalized Ricci flow
\item Perelman's monotonicity formulas
\item Convergence results
\end{itemize}

\section{Technical Improvements}

\begin{itemize}
\item Full tangent bundle implementation
\item Complete Riemann tensor computation
\item Time-dependent Riemannian metrics
\item Manifold calculus infrastructure
\end{itemize}
