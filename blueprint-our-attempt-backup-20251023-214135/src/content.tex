\chapter{Introduction}

This blueprint documents the formalization of Ricci Flow theory in Lean 4.
Ricci Flow is a fundamental geometric evolution equation introduced by Richard Hamilton,
which has profound applications including the proof of the Poincaré conjecture.

\chapter{Basic Lemmas}
\label{chap:basic}

This chapter contains fundamental lemmas about real numbers and topology
that are used throughout the formalization.

\section{Properties of Positive Reals}

\begin{lemma}[Positive multiplication]
\label{lem:pos_mul_pos}
\lean{RicciFlow.pos_mul_pos}
\leanok
For positive real numbers $a$ and $b$, their product $a \cdot b$ is positive.
\end{lemma}

\begin{proof}
\leanok
This follows directly from Mathlib's \texttt{mul\_pos} theorem.
\end{proof}

\begin{lemma}[Square of nonzero is positive]
\label{lem:square_pos_of_ne_zero}
\lean{RicciFlow.square_pos_of_ne_zero}
\leanok
If $x \neq 0$, then $x^2 > 0$.
\end{lemma}

\begin{proof}
\leanok
This is a consequence of the fact that squares of nonzero numbers are positive,
formalized using Mathlib's \texttt{sq\_pos\_of\_ne\_zero}.
This property is fundamental for proving positive-definiteness of the metric tensor.
\end{proof}

\begin{lemma}[Existence of positive real]
\label{lem:exists_pos_real}
\lean{RicciFlow.exists_pos_real}
\leanok
There exists a real number $t$ such that $t > 0$.
\end{lemma}

\begin{proof}
\leanok
Constructively, we exhibit $t = 1$ and verify $1 > 0$ using \texttt{norm\_num}.
\end{proof}

\begin{lemma}[Inverse of positive is positive]
\label{lem:inv_pos_of_pos}
\lean{RicciFlow.inv_pos_of_pos}
\leanok
If $x > 0$, then $x^{-1} > 0$.
\end{lemma}

\begin{proof}
\leanok
This is \texttt{inv\_pos.mpr} from Mathlib.
Used when computing scalar curvature $R = g^{ij} \mathrm{Ric}_{ij}$ where
we need the inverse metric $g^{ij}$ to preserve positivity properties.
\end{proof}

\section{Topological Properties}

\begin{lemma}[Continuity equivalence]
\label{lem:continuous_at_iff}
\lean{RicciFlow.continuousAt_iff_continuousWithinAt}
\leanok
A function $f$ is continuous at $x$ if and only if it is continuous within
the universal set at $x$.
\end{lemma}

\begin{proof}
\leanok
This follows from \texttt{continuousWithinAt\_univ} in Mathlib.
Useful when working with smooth functions in local coordinate charts.
\end{proof}

\chapter{Riemannian Manifolds}
\label{chap:riemannian}

We introduce the structures of Riemannian geometry.

\begin{definition}[Riemannian Metric]
\label{def:riemannian_metric}
\lean{RicciFlow.RiemannianMetric}
\leanok
\uses{lem:pos_mul_pos, lem:square_pos_of_ne_zero}
A Riemannian metric on a manifold $M$ is a smooth assignment to each point $x \in M$
of a positive-definite symmetric bilinear form $g_x : T_xM \times T_xM \to \mathbb{R}$.

Formally, it is a structure containing:
\begin{itemize}
\item A function \texttt{toFun} : $M \to (V \to V \to \mathbb{R})$ where $V$ represents the tangent space
\item \textbf{Symmetry}: $g_x(v, w) = g_x(w, v)$ for all $x, v, w$
\item \textbf{Positive-definiteness}: $g_x(v, v) > 0$ for all $x$ and all $v \neq 0$
\end{itemize}

\textbf{Implementation note}: The current version uses a type parameter $V$ for the abstract
tangent space. The full implementation should use the actual tangent bundle $TM$.
\end{definition}

\begin{definition}[Inner Product]
\label{def:inner_product}
\lean{RicciFlow.innerProduct}
\leanok
\uses{def:riemannian_metric}
Given a Riemannian metric $g$ on $M$, the inner product of two tangent vectors
$v, w \in T_xM$ at point $x$ is defined as:
\[ \langle v, w \rangle_x := g_x(v, w) \]
\end{definition}

\begin{definition}[Norm Squared]
\label{def:norm_sq}
\lean{RicciFlow.normSq}
\leanok
\uses{def:riemannian_metric}
The squared norm of a tangent vector $v \in T_xM$ at point $x$ is:
\[ \|v\|^2 := g_x(v, v) \]

The actual norm is $\|v\| = \sqrt{\|v\|^2}$.
\end{definition}

\begin{lemma}[Inner product symmetry]
\label{lem:inner_product_symm}
\lean{RicciFlow.innerProduct_symm}
\leanok
\uses{def:inner_product}
For any tangent vectors $v, w$ at point $x$:
\[ \langle v, w \rangle_x = \langle w, v \rangle_x \]
\end{lemma}

\begin{proof}
\leanok
\uses{def:riemannian_metric}
This follows directly from the symmetry axiom of the Riemannian metric.
\end{proof}

\begin{lemma}[Norm squared positivity]
\label{lem:norm_sq_pos}
\lean{RicciFlow.normSq_pos}
\leanok
\uses{def:norm_sq}
For any nonzero tangent vector $v$ at point $x$:
\[ \|v\|^2 > 0 \]
\end{lemma}

\begin{proof}
\leanok
\uses{def:riemannian_metric}
This follows from the positive-definiteness axiom of the Riemannian metric.
\end{proof}

\chapter{Ricci Curvature}
\label{chap:ricci}

The Ricci tensor encodes information about the curvature of a Riemannian manifold.

\begin{definition}[Ricci Tensor]
\label{def:ricci_tensor}
\lean{RicciFlow.RicciTensor}
\leanok
\uses{def:riemannian_metric}
The Ricci tensor $\mathrm{Ric}$ is obtained by contracting the Riemann curvature tensor:
\[ \mathrm{Ric}_{ij} = R^k_{ikj} = g^{kl} R_{likj} \]
where $R_{ijkl}$ is the Riemann curvature tensor and $g^{kl}$ is the inverse metric.

\textbf{Current implementation}: This is a simplified structure that stores the trace value
(scalar curvature) directly. The full implementation should compute this from the Riemann tensor.
\end{definition}

\begin{definition}[Scalar Curvature]
\label{def:scalar_curvature}
\lean{RicciFlow.scalarCurvature}
\leanok
\uses{def:ricci_tensor, lem:inv_pos_of_pos}
The scalar curvature $R$ is the trace of the Ricci tensor with respect to the metric:
\[ R = g^{ij} \mathrm{Ric}_{ij} \]

\textbf{Geometric interpretation}:
\begin{itemize}
\item $R > 0$: The manifold curves inward (like a sphere)
\item $R < 0$: The manifold curves outward (like hyperbolic space)
\item $R = 0$: The manifold is locally flat (like Euclidean space)
\end{itemize}
\end{definition}

\begin{lemma}[Scalar curvature computation]
\label{lem:scalar_curvature_eq}
\lean{RicciFlow.scalarCurvature_eq_traceValue}
\leanok
\uses{def:scalar_curvature}
In our current implementation, the scalar curvature equals the stored trace value:
\[ R = \mathrm{Ric}.\mathrm{traceValue} \]
\end{lemma}

\begin{proof}
\leanok
This is definitional in our simplified implementation.
\end{proof}

\chapter{Ricci Flow}
\label{chap:flow}

The Ricci flow is a geometric evolution equation that deforms a Riemannian metric.

\begin{definition}[Ricci Flow Equation]
\label{def:ricci_flow_equation}
\lean{RicciFlow.ricci_flow_equation}
\uses{def:riemannian_metric, def:ricci_tensor}
The Ricci flow equation is a parabolic PDE for a time-dependent Riemannian metric $g(t)$:
\[ \frac{\partial g_{ij}}{\partial t} = -2 \mathrm{Ric}_{ij} \]

This equation describes how the metric evolves to become more uniform over time.

\textbf{Current status}: Declared as an axiom. The full definition requires
setting up the theory of time-dependent metrics and their derivatives.
\end{definition}

\begin{theorem}[Short-Time Existence]
\label{thm:short_time_existence}
\lean{RicciFlow.short_time_existence}
\uses{def:ricci_flow_equation, lem:exists_pos_real}
For any smooth compact Riemannian manifold $(M, g_0)$, there exists $T > 0$
and a smooth solution $g : [0, T) \to \mathrm{Met}(M)$ to the Ricci flow equation
with initial condition $g(0) = g_0$.
\end{theorem}

\begin{proof}
\textbf{Current status}: Declared as \texttt{sorry}.

The full proof requires three main steps:
\begin{enumerate}
\item \textbf{DeTurck's trick}: Introduce the Ricci-DeTurck flow
  \[ \frac{\partial g_{ij}}{\partial t} = -2 \mathrm{Ric}_{ij} + \mathcal{L}_W g_{ij} \]
  where $W$ is chosen to break diffeomorphism invariance.

\item \textbf{Parabolic theory}: Apply standard short-time existence theory for
  parabolic PDEs to the Ricci-DeTurck flow. This is the main analytic work.

\item \textbf{Pullback}: Show that the Ricci-DeTurck solution can be pulled back
  by a family of diffeomorphisms to obtain a Ricci flow solution.
\end{enumerate}

Key references:
\begin{itemize}
\item Hamilton, ``Three-manifolds with positive Ricci curvature" (1982)
\item DeTurck, ``Deforming metrics in the direction of their Ricci tensors" (1983)
\item Topping, ``Lectures on the Ricci Flow" (2006)
\end{itemize}
\end{proof}

\chapter{Future Work}

\section{Planned Extensions}

\begin{itemize}
\item Complete the proof of short-time existence
\item Formalize the maximum principle for Ricci flow
\item Implement normalized Ricci flow
\item Develop the theory of ancient solutions
\item Formalize Perelman's monotonicity formulas
\item Prove convergence results for special geometries
\end{itemize}

\section{Technical Improvements}

\begin{itemize}
\item Replace the simplified tangent space representation with proper tangent bundle theory
\item Implement full Riemann curvature tensor computation
\item Add support for time-dependent Riemannian metrics
\item Develop calculus on manifolds infrastructure
\end{itemize}
